\documentclass[a4paper]{article}
%\usepackage[T1]{fontenc}
\usepackage[utf8]{inputenc}
\usepackage{lmodern}

\usepackage[ngerman]{babel}
 
\author{Tim B. Herbstrith}
\title{\LaTeX-Tutorium}
\date{Oktober 2015}

\usepackage{biblatex}
\addbibresource{sources.bib}

\usepackage{hyperref}
\begin{document}
	\maketitle
	
	Dieses Tutorium basiert auf \cite{l2short}.
	Eine ausführlichere Einleitung zu \LaTeX{} bietet \cite{texbuch}.\\
	
	Als Beispiel für einen exzellenten Textsatz wird die 
	digitalisierte 42-zeilige Gutenberg-Bibel \cite{gutenberg} 
	aus dem Bestand der Österreichischen Nationalbibliothek verwendet. 
	Die Kodierungstabellen stammen aus \cite{encodings}.\\
	
	Die einzelnen Dateien sind in meinem
	\href{https://github.com/tim6her/BAS1}{Git-Repository}
	unter der 
	\href{http://www.gnu.org/licenses/gpl.html}{GNU General Public License}
	verfügbar.
	
	\printbibliography
\end{document}